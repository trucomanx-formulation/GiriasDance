\documentclass[a4paper,10pt]{article}
\usepackage[utf8]{inputenc}
\usepackage{comment}

\usepackage[
backend=biber,
style=alphabetic,
sorting=ynt
]{biblatex}
\addbibresource{bibliography.bib}   %>>>> bibliography data in bibliography.bib
%\bibliography{bibliography}   %>>>> bibliography data in bibliography.bib
%\bibliographystyle{spiebib}   %>>>> makes bibtex use spiebib.bst


%%%%%%%%%%%%%%%%%%%%%%%%%%%%%%%%%%%%%%%%%%%%%%%%%%%%%%%%%%%%%%%%%%%%%%%%%%%%%%%%
\usepackage{url}
%
\usepackage[breaklinks]{hyperref}
\usepackage{breakurl}

%%%%%%%%%%%%%%%%%%%%%%%%%%%%%%%%%%%%%%%%%%%%%%%%%%%%%%%%%%%%%%%%%%%%%%%%%%%%%%%%
\usepackage{graphicx}
\usepackage{subcaption}
%%%%%%%%%%%%%%%%%%%%%%%%%%%%%%%%%%%%%%%%%%%%%%%%%%%%%%%%%%%%%%%%%%%%%%%%%%%%%%%%

\usepackage[svgnames]{xcolor} % Enabling colors by their 'svgnames'

\usepackage{amsmath}
\usepackage{amsfonts}
\usepackage{amssymb}


%%%%%%%%%%%%%%%%%%%%%%%%%%%%%%%%%%%%%%%%%%%%%%%%%%%%%%%%%%%%%%%%%%%%%%%%%%%%%%%%
\usepackage{amsthm}
\newtheorem{mydefinition}{Definição}

%%%%%%%%%%%%%%%%%%%%%%%%%%%%%%%%%%%%%%%%%%%%%%%%%%%%%%%%%%%%%%%%%%%%%%%%%%%%%%%%

%opening
%\title{Neologismo de sentido na didática na dança de salão no Brasil}
\title{Gírias na didática na dança de salão no Brasil}
\author{Fernando Pujaico Rivera, -------- -------- --------}

%%%%%%%%%%%%%%%%%%%%%%%%%%%%%%%%%%%%%%%%%%%%%%%%%%%%%%%%%%%%%%%%%%%%%%%%%%%%%%%%
%%%%%%%%%%%%%%%%%%%%%%%%%%%%%%%%%%%%%%%%%%%%%%%%%%%%%%%%%%%%%%%%%%%%%%%%%%%%%%%%
%%%%%%%%%%%%%%%%%%%%%%%%%%%%%%%%%%%%%%%%%%%%%%%%%%%%%%%%%%%%%%%%%%%%%%%%%%%%%%%%
%%%%%%%%%%%%%%%%%%%%%%%%%%%%%%%%%%%%%%%%%%%%%%%%%%%%%%%%%%%%%%%%%%%%%%%%%%%%%%%%
\begin{document}

\maketitle

\begin{abstract}
Condução
\end{abstract}


%%%%%%%%%%%%%%%%%%%%%%%%%%%%%%%%%%%%%%%%%%%%%%%%%%%%%%%%%%%%%%%%%%%%%%%%%%%%%%%%
%%%%%%%%%%%%%%%%%%%%%%%%%%%%%%%%%%%%%%%%%%%%%%%%%%%%%%%%%%%%%%%%%%%%%%%%%%%%%%%%
%%%%%%%%%%%%%%%%%%%%%%%%%%%%%%%%%%%%%%%%%%%%%%%%%%%%%%%%%%%%%%%%%%%%%%%%%%%%%%%%

\section{Fundamento teórico}

Dado que no Brasil, no existe um dicionario oficial, 
que regule ou marque o rumbo do significado das palavras;
acho necessário, antes de continuar com as explicações, 
e para evitar incorrer, não intencionadamente, em mal entendidos;
emoldurar, o uso de alguns termos, nas acepções das palavras propostas por alguns autores.
\begin{mydefinition}[Gíria:] 
%\index{Gíria}
\label{def:Giria}
Podem ser achadas as seguintes acepções no Dicionário Priberam da Língua Portuguesa \cite{priberamgiria}:
\begin{itemize}
\item Linguagem característica de um grupo profissional ou sociocultural, é equivalente ao termo JARGÃO.
\item Linguagem usada por determinado grupo, 
geralmente incompreensível para quem não pertence ao grupo e que serve também como meio de realçar a sua especificidade.
\end{itemize}
\end{mydefinition}

\begin{mydefinition}[Neologismo:] 
%\index{Neologismo}
\label{def:Neologismo}
Podem ser achadas as seguintes acepções no Dicionário Priberam da Língua Portuguesa \cite{priberamneologismo}:
\begin{itemize}
\item Palavra nova, ou acepção nova de uma palavra já existente na língua.
\item Emprego de palavras novas ou de novas acepções.
\end{itemize}
\end{mydefinition}

\begin{mydefinition}[Neologismo semântico:] 
%\index{Neologismo!Neologismo semântico}
\label{def:NeologismoSemantico}
Um neologismo semântico ou neologismo de significado é caraterizado pela modificação do significado de uma palavra já existente na língua;
as gírias em muitas ocasiões constituem exemplos de neologismos semânticos, 
pois algumas gírias dão novos sentidos a palavras já usadas no vocabulário formal \cite[pp. 82-83]{correalingua}.
\end{mydefinition}


\begin{mydefinition}[Homonímia:] 
%\index{Homonímia}
\label{def:Homonimia}
~

\begin{itemize}
\item Relação entre palavras que têm sentidos e origens diferentes
 mas que se escrevem e pronunciam da mesma maneira \cite{priberamhomonimia}.

\end{itemize}
\end{mydefinition}

%\begin{comment}
%\end{comment}
\section{Homonímia vs. neologismo semântico}
Aqui analisaremos os casos onde temos um novo termo na linguagem popular,
porem a palavra usada para designar o englobar a ideia desse novo termo,
já existe na linguagem formal, nesse ponto cabe nos perguntar, 
estamos frente a uma homonimia ou a um neologismo semântico?

A língua dispõe de mecanismos controladores da proliferação de homonimia,
como uma forma de preservara a eficacia e a clareza da comunicação \cite[pp. 13]{pilla2002neologismos}.

A homonimia é quando esse novo termo engloba um conceito conhecido \cite[pp. 13]{pilla2002neologismos}.

e o neologismo semântico acontece quando esse novo termo engloba um  conceito novo \cite[pp. 13]{pilla2002neologismos}.

%%%%%%%%%%%%%%%%%%%%%%%%%%%%%%%%%%%%%%%%%%%%%%%%%%%%%%%%%%%%%%%%%%%%%%%%%%%%%%%%
%%%%%%%%%%%%%%%%%%%%%%%%%%%%%%%%%%%%%%%%%%%%%%%%%%%%%%%%%%%%%%%%%%%%%%%%%%%%%%%%
%%%%%%%%%%%%%%%%%%%%%%%%%%%%%%%%%%%%%%%%%%%%%%%%%%%%%%%%%%%%%%%%%%%%%%%%%%%%%%%%
\section{Não recomendadas na dança de salão}

\subsection{Neologismo+Giria: contratempo}
Eu vejo que muitos profissionais na dança usam na sua didática a palavra ``contratempo'', 
de formas diferentes, 
alguns usam a palavra contratempo para definir uma distribuição de tempos como: 
$\frac{T}{2}+\frac{T}{2}+T$ (é dizer ``chic-chic tum''), 
e outros como: $\frac{T}{2}$ (é dizer ``chic''); 
sendo que o termo contratempo, 
já está bem definido na literatura \cite{priberamcontratempo} e em especial na relativa à música \cite[pp. 46]{alves2004teoria}.
Isto a meu parecer cria um problema na didática no ensino da dança. 
Antes de expor completamente meu ponto de vista, gostaria relatar primeiro uma experiencia pessoal que ilustre a problemática: 
\begin{quotation}
O professor de dança nos queria ensinar algo de música e musicalidade, 
porem existindo vários profissionais da música, bem qualificados e disponíveis gratuitamente na faculdade; 
o professor de dança não conseguia um deles pra nos ensinar,
porque ele precisava que o professor de música fosse também da área da dança. 
Mais porque? Era porque esse professor de música, 
para não criar conflitos no entendimento das explicações, 
deveria conhecer as gírias que alguns profissionais da dança usam na sua particular visão e didática. 
\end{quotation}
Assim, entre os termos usados, com significados trocados, estão a palavra contratempo, 
a forma de contar os tempos no compasso, a definição de sincopa, como contar a frase musical, etc. 
Porem gostaria ocupar-me aqui só do caso do contratempo. 

Minha argumentação vai em contra do uso da palavra contratempo no sentido de uma distribuição de tempos, 
dado que no primeiro lugar, já existem termos para isto; como: ``chic-chic tum'', 
``rápido-rápido lento'' ou ``quick-quick slow'' e ``triple step'' da notação usada em inglês.
por outro lado, sobrecarregar o significado de uma palavra, com um sentido diferente ao original, 
em dois âmbitos tão próximos como a dança e a música; cria uma separação, 
entre os conhecimentos dos profissionais destes dois âmbitos, o que se reflete no avanço da pedagogia na dança; 
pois isola aos profissionais, a um mundo particular criado por eles mesmos.  
A musica é uma linguagem, e criar uma linguagem paralela, por mais simplificada que seja, 
pode solucionar problemas a curto prazo, porem isola de um conhecimento global e maior ao longo prazo.
Gostaria ressaltar também, que o problema, de que os profissionais da musica e da dança não falam o mesmo idioma, 
acontece em alguns profissionais da dança de salão no Brasil, porem com preocupante frequência. 
Por outro lado, em estilos de dança com projeção internacional como a bachata e a salsa, 
é mais frequente achar, nos profissionais da dança, pessoas que são suficientemente rigorosos  no uso de alguns termos musicais, 
e isto contribuiu à globalização dessas danças, pois permite a todos no mundo falar um mesmo idioma.

Finalmente, gostaria rebater o argumento de que o contratempo na dança ``é outro'' contratempo, 
diferente ao musical, pois como falei antes, 
ter palavras com significados diferentes em âmbitos tão próximos é inconveniente; 
e mesmo ignorando este ponto, a afirmação de que o contratempo na dança é outro, é falso, 
pois já existem danças de salão que usam a palavra  contratempo num sentido mais correto. 
Por exemplo: no son cubano \cite[pp. 36-37]{borges2012historia}. 
Ali distinguem, dançar a tempo (dançar com o movimento principal no tempo forte) 
e dançar a contratempo (dançar com o movimento principal em contra do tempo forte). 
É mais, não precisamos ir ate esse estilo de dança, para ver um uso correto da palavra; 
no passo: caminhada a contratempo, na samba de gafieira, já podemos ver este uso correto. 
Pois a pessoa que faz este passo, 
inicia dançando no tempo forte (passo longo (principal) do do ``chic-chic tum'' no tempo forte) e ao finalizar este passo, 
termina-se dançando em contra do tempo forte (a contratempo), é dizer o passo longo no tempo fraco; 
pois lembremos que a caminhadas a contratempo usa uma distribuição de tempos: $\frac{T}{2}+\frac{T}{2}+T+T$  (``chic-chic tum tum''). 
É dizer um compasso binário e médio, o que ocasiona a desfasagem do tempo forte.

Eu não estou argumentando em contra da ideia do contratempo na dança, como conceito. 
Só opino, que deve-se mudar a palavra por outra, por exemplo usar, 
rápido-rápipo lento; ou como outros profissionais usam em gafieira, 
chic-chic tum, ou tic-tic tum \cite[pp. 146]{perna2002samba}. 
Se mudar a palavra contratempo, por alguns dos anteriores exemplos, é muito difícil para algumas pessoas; 
gostaria propor o uso da palavra "contra-passo", que ao igual que contratempo, tem 4 silabas, 
e pode ser usado facilmente com o mesmo proposito, sem sobrecarregar a palavra contratempo com significados distintos.

 %[OK]

%\input{giria-ritmo}

 
%%%%%%%%%%%%%%%%%%%%%%%%%%%%%%%%%%%%%%%%%%%%%%%%%%%%%%%%%%%%%%%%%%%%%%%%%%%%%%%%
%%%%%%%%%%%%%%%%%%%%%%%%%%%%%%%%%%%%%%%%%%%%%%%%%%%%%%%%%%%%%%%%%%%%%%%%%%%%%%%%
%%%%%%%%%%%%%%%%%%%%%%%%%%%%%%%%%%%%%%%%%%%%%%%%%%%%%%%%%%%%%%%%%%%%%%%%%%%%%%%%
\section{Conclusões}
Nas nossas pesquisas sobre 
%%%%%%%%%%%%%%%%%%%%%%%%%%%%%%%%%%%%%%%%%%%%%%%%%%%%%%%%%%%%%%%%%%%%%%%%%%%%%%%%
%%%%%%%%%%%%%%%%%%%%%%%%%%%%%%%%%%%%%%%%%%%%%%%%%%%%%%%%%%%%%%%%%%%%%%%%%%%%%%%%
%%%%%%%%%%%%%%%%%%%%%%%%%%%%%%%%%%%%%%%%%%%%%%%%%%%%%%%%%%%%%%%%%%%%%%%%%%%%%%%%
%%%%%%%%%%%%%%%%%%%%%%%%%%%%%%%%%%%%%%%%%%%%%%%%%%%%%%%%%%%%%%%%%%%%%%%%%%%%%%%%

\medskip
 
\printbibliography

\end{document}
