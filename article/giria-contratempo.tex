\subsection{Contratempo como neologismo semântico}
Eu vejo que muitos profissionais na dança usam na sua didática a palavra ``contratempo'' como neologismo semântico, 
atribuindo-lhe diferentes significados, 
alguns usam a palavra contratempo para definir uma distribuição de tempos como: 
$\frac{T}{2}+\frac{T}{2}+T$ (é dizer um ``Tic-tic tum'' --- neologismo fonológico), 
e outros como: $\frac{T}{2}$ (é dizer um ``Tic'' --- neologismo fonológico); 
sendo que o termo contratempo, 
já está bem definido na literatura \cite{priberamcontratempo} e em especial na relativa à música \cite[pp. 46]{alves2004teoria}.
Isto a meu parecer cria um problema na didática no ensino da dança. 
Antes de expor completamente meu ponto de vista, gostaria relatar primeiro um exemplo que ilustre a problemática: 
\begin{quotation}
O professor de dança nos queria ensinar algo de música e musicalidade, 
porém existindo vários profissionais e estudiosos da música, 
bem qualificados e disponíveis gratuitamente ao redor de nosso local de trabalho; 
o professor de dança não conseguia um deles pra nos ensinar.

O problema era que o professor de dança precisava que o professor de música fosse também da área da dança. 
Mas porque? Se estando um professor da área da música e outro da dança seria mais que suficiente.
Isto era assim porque era necessário que o professor de música
conheça as gírias (neologismos semânticos) que alguns profissionais da dança usam na sua particular visão e didática, 
para não criar conflitos nos alunos no entendimento das explicações. 
\end{quotation}
Assim, entre os termos usados, com significados trocados, estão a palavra contratempo, 
a forma de contar os tempos no compasso, a definição de sincopa, como contar a frase musical,
o uso da palavra ritmo, etc. 
porém gostaria ocupar-me aqui só do caso do contratempo. 

Minha argumentação vai em contra do uso da palavra contratempo como neologismo semântico, 
sendo usada desta forma por alguns professores de dança
para indicar uma distribuição de tempos; pois este uso vai em contradição do seu significado musical. 
Também é importante ressaltar que existem termos alternativos a este neologismo semântico, 
que evitam o conflito de significados; 
assim temos os termos: ``Tic-tic tum'' (neologismo fonológico), 
``rápido-rápido lento'' (neologismo sintático) ou ``quick-quick slow'' (neologismo sintático) e ``triple step'' da notação usada em inglês.

Sobrecarregar o significado de uma palavra, com um sentido diferente aos já usados na linguagem formal, 
em dois âmbitos tão próximos como a dança e a música; cria uma separação, 
entre os conhecimentos dos profissionais destes dois âmbitos, o que se reflete no avanço da pedagogia na dança; 
pois isola aos profissionais, a um mundo particular criado por eles mesmos.  
A musica é uma linguagem, e criar uma linguagem paralela, por mais simplificada que seja, 
pode solucionar problemas a curto prazo, porém isola de um conhecimento global e maior ao longo prazo.

Gostaria ressaltar que o problema de que os profissionais da música e da dança de salão no Brasil não falem um mesmo idioma musical, 
somente acontece com alguns profissionais da dança, porém se observa com preocupante frequência. 
Por outro lado, é possível observar que em estilos de dança com projeção internacional, 
como por exemplo a bachata ou a salsa, 
é mais frequente observar, nos profissionais da dança, 
pessoas que são suficientemente rigorosas no uso de alguns termos musicais \FALTAREFERENCIA,
isto se evidencia pela variedade de artigos que relacionam dança e música \cite{10.1093.mts.mty033} \cite{10.3389.fnhum.2016.00064}, 
e isto contribuiu à globalização dessas danças, pois permite que todos no mundo da dança falem um mesmo idioma musical.
Por exemplo se vemos os resultado do Campeonato Mundial de Salsa (do inglês ``World Salsa Championships'')
do ano 2016 \cite{WSC2016} observamos uma grande variedade de nacionalidades dos participantes
nos primeiros lugares :
\begin{inparaitem}
\item Brasil, Peru e Panamá na ``On 1 Division''.
\item Itália e México na ``On 2 Division''.
\item Argentina, Chile e Colômbia na ``Cabaret Division''.
\item Porto Rico e USA na ``Famale Same Gender Division''.
\item Colômbia e México na ``Male Same Gender Division''.
\item USA e Equador na ``Team Division''.
\end{inparaitem}

Finalmente, gostaria rebater o argumento de que o contratempo na dança ``é simplesmente outro contratempo'', 
diferente ao musical, pois como falei antes, 
ter palavras com significados diferentes em âmbitos tão próximos é inconveniente; 
e mesmo ignorando este ponto, a afirmação de que o contratempo na dança é outro, é falso, 
pois já existem danças de salão que usam a palavra  contratempo num sentido mais correto. 
Por exemplo: no son cubano \cite[pp. 36-37]{borges2012historia}, 
pois ali distinguem entre dançar a tempo (dançar com o movimento principal no tempo forte) 
e dançar a contratempo (dançar com o movimento principal em contra do tempo forte). 
É mais, não precisamos ir ate esse estilo de dança, para ver um uso correto da palavra contratempo; 
no passo: caminhada a contratempo, na samba de gafieira, já podemos ver este uso correto. 
Pois a pessoa que faz este passo, 
inicia dançando no tempo forte, ex: passo longo e principal do do ``Tic-tic tum'' no tempo forte, e 
ao finalizar este passo, 
termina-se dançando em contra do tempo forte (a contratempo), é dizer o passo longo no tempo fraco; 
pois devemos lembrar que a caminhada a contratempo usa uma distribuição de tempos: $\frac{T}{2}+\frac{T}{2}+T+T$  (``Tic-tic tum tum''). 
É dizer um compasso binário e médio, o que ocasiona a desfasagem do tempo forte.

Eu não estou argumentando em contra do significado do contratempo na dança
e sim do significante. 
Opino, que deve-se mudar a palavra por outra, por exemplo usar, 
tic-tic tum como fazem alguns profissionais do samba de gafieira \cite[pp. 146]{perna2002samba}.
Ou também, se mudar a palavra contratempo é muito difícil para algumas pessoas; 
gostaria propor o uso da palavra "contra-passo", que ao igual que contratempo, tem 4 silabas, 
e pode ser usado facilmente com o mesmo proposito, sem sobrecarregar a palavra contratempo com significados distintos.

