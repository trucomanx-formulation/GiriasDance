\section{Homonímia vs. neologismo semântico}
Nesta seção serão analisados os casos nos quais temos uma palavra que já existe na linguagem formal,
porém que na linguagem popular ganha um novo significado. Nesse ponto cabe nos perguntar, 
estamos frente a uma homonimia ou a um neologismo semântico?

O caso do paragrafo anterior poderia a primeira vista ser emoldurado nas Definições \ref{def:NeologismoSemantico} e \ref{def:Homonimia},
porém entre estas duas existe uma sutil diferença, a homonímia se refere a palavras e significados existentes na linguagem formal, 
já se o significado da palavra é novo e foi introduzido ao linguagem popular estamos frente a um neologismo semântico.
Por exemplo, se selecionamos a palavra sapato, que já existe na linguagem formal,
e definimos que a partir de agora usaremos a palavra sapato para designar a um amigo que nos acompanha a todos lados,
com frases como: ``Raul é meu sapato, ele está presente em todas minhas viagens''.
Nesse caso não estamos criando homonímia e sim um neologismo semântico,
pois estes neologismos acontecem quando uma nova realidade concreta ou abstrata pode ser elaborada por médio de uma palavra já existente \cite[pp. 13]{pilla2002neologismos}.

Um neologismo, mediante um processo longo no tempo, pode passar da linguagem popular à linguagem formal. 
Porém a formalização de um neologismo semântico 
é um evento que ocorre com pouca frequência, pois
a língua dispõe de mecanismos controladores da proliferação de homonimia,
para preservar a eficacia e a clareza da comunicação \cite[pp. 13]{pilla2002neologismos}.
Nesse sentido podemos ler no artigo ``Ambigüidade gerada pela homonímia: revisitação teórica, linhas limítrofes com a polissemia e proposta de critérios distintivos'' \cite{ZAVAGLIA2003} \cite{ullmann1973semantica}:
\begin{quotation}
``Apesar de a homonímia ser muito menos comum e complexa do que a polissemia, seus efeitos podem ser tão graves quanto ou até mesmo mais contundentes.''
\end{quotation} 


