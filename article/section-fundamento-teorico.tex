\section{Fundamento teórico}

Dado que no Brasil não existe um dicionario oficial
que regule ou marque o rumbo do significado das palavras;
é necessário, antes de continuar com as explicações, 
e para evitar incorrer, não intencionadamente, em mal entendidos;
emoldurar, o uso de alguns termos, nas acepções das palavras propostas por alguns autores.
\begin{mydefinition}[Gíria:] 
%\index{Gíria}
\label{def:Giria}
Podem ser achadas as seguintes acepções no Dicionário Priberam da Língua Portuguesa \cite{priberamgiria}:
\begin{itemize}
\item Linguagem característica de um grupo profissional ou sociocultural, é equivalente ao termo JARGÃO.
\item Linguagem usada por determinado grupo, 
geralmente incompreensível para quem não pertence ao grupo e que serve também como meio de realçar a sua especificidade.
\end{itemize}
\end{mydefinition}

\begin{mydefinition}[Neologismo:] 
%\index{Neologismo}
\label{def:Neologismo}
Podem ser achadas as seguintes acepções no Dicionário Priberam da Língua Portuguesa \cite{priberamneologismo}:
\begin{itemize}
\item Palavra nova, ou acepção nova de uma palavra já existente na língua.
\item Emprego de palavras novas ou de novas acepções.
\end{itemize}
\end{mydefinition}

\begin{mydefinition}[Neologismo fonológico:] 
%\index{Neologismo!Neologismo fonológico}
\label{def:NeologismoFonologico}
Um neologismo fonológico é caraterizado pela creação de uma palavra cujo significado é inedito,
é dizer que não está formado em base a alguma palavra existente
\cite[pp. 81-82]{correalingua}, exemplo: a chiforinfula do Chaves.
\end{mydefinition}

\begin{mydefinition}[Neologismo sintático:] 
%\index{Neologismo!Neologismo fonológico}
\label{def:NeologismoFonologico}
Um neologismo sintático ou neologismo de forma é caraterizado pela creação de uma palavra 
mediante a combinação de palavras existentes, exemplo: Miniserie, bioterra
\cite[pp. 82]{correalingua}.
\end{mydefinition}

\begin{mydefinition}[Neologismo semântico:] 
%\index{Neologismo!Neologismo semântico}
\label{def:NeologismoSemantico}
Um neologismo semântico ou neologismo de significado é caraterizado pela modificação do significado de uma palavra já existente na língua;
as gírias em muitas ocasiões constituem exemplos de neologismos semânticos, 
pois algumas gírias dão novos sentidos a palavras já usadas no vocabulário formal \cite[pp. 82-83]{correalingua}.
\end{mydefinition}


\begin{mydefinition}[Homonímia:] 
%\index{Homonímia}
\label{def:Homonimia}
Relação entre palavras que têm sentidos e origens diferentes
 mas que se escrevem e pronunciam da mesma maneira \cite{priberamhomonimia}.
\end{mydefinition}

